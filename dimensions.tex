\documentclass{book}
\usepackage[brazilian]{babel}
\usepackage[utf8]{inputenc}
\usepackage[T1]{fontenc}
\usepackage{multirow}

\usepackage{amsmath}
\usepackage{graphicx}
\graphicspath{ {Figures/} }
\usepackage{float}
\usepackage[dvipsnames]{xcolor}
\usepackage{tcolorbox}

\usepackage{natbib,stfloats}

\tcbuselibrary{skins,breakable}
\usetikzlibrary{shadings,shadows}
\usepackage{rotating}

\usepackage{tikz}
\usetikzlibrary{snakes}
\usetikzlibrary{patterns}

\usepackage{colortbl}
\newcolumntype{a}{>{\columncolor{yellow}}l}
\newcolumntype{b}{>{\columncolor{blue}}l}
\newcolumntype{d}{>{\columncolor{green}}l}

\usepackage{fancyhdr}
\pagestyle{fancy}

\fancyhead{}
\fancyhead[RO]{\textsl{\rightmark}} 
\fancyhead[LE]{\textsl{\leftmark}} 

\fancyfoot{} 
\fancyfoot[L]{T.B. Fraga. thoughts: dimensions of our universe.}
\fancyfoot[R]{\thepage}

\definecolor{cambridgeblue}{rgb}{0.64, 0.76, 0.68}
\definecolor{camel}{rgb}{0.76, 0.6, 0.42}
\definecolor{camouflagegreen}{rgb}{0.47, 0.53, 0.42}
\definecolor{desertsand}{rgb}{0.93, 0.79, 0.69}

\newenvironment{exerciseblock}[1]{%
    \tcolorbox[beamer,%
    noparskip,breakable,
    colback=camel!20,colframe=camouflagegreen,%
    colbacklower=camouflagegreen!75!camel!20,%
    title=#1]}%
    {\endtcolorbox}

\newenvironment{exempleblock}[1]{%
    \tcolorbox[beamer,%
    noparskip,breakable,
    colback=camel!20,colframe=Brown!70,%
    colbacklower=camouflagegreen!75!camel!20,%
    title=#1]}%
    {\endtcolorbox}

\newenvironment{algorithmblock}[1]{%
    \tcolorbox[beamer,%
    noparskip,breakable,
    colback=camel!20,colframe=Brown!40,%
    colbacklower=camouflagegreen!75!camel!20,%
    title=#1]}%
    {\endtcolorbox}

\newenvironment{dedication}
  {\clearpage           % we want a new page
   \thispagestyle{empty}% no header and footer
   \vspace*{\stretch{1}}% some space at the top 
   \itshape             % the text is in italics
   \raggedleft          % flush to the right margin
  }
  {\par % end the paragraph
   \vspace{\stretch{3}} % space at bottom is three times that at the top
   \clearpage           % finish off the page
  }

% para construção de diagramas

\tikzstyle{place}=[circle,draw=camouflagegreen!80,fill=desertsand!20,thick]
\tikzstyle{Ret}=[rectangle,draw=camouflagegreen!80,fill=desertsand!20,thick]
\tikzstyle{RoundRet}=[rectangle,draw=camouflagegreen!80,fill=desertsand!20,thick, rounded corners]

\begin{document}

I recently saw a video about string theory. This video explained that there was a theory that explained all the fundamental forces. But that theory started from the premise that our universe was composed of at least 10 dimensions (9 spatial and 1 temporal). \\

Then I saw another video by Daniel Garbin on value engineering, in which he presents a chart for product analysis (Fig. \ref{fig:productAnalisys}). 

\begin{figure}[h]
	\begin{center}
	    \caption{Graph for product analysis. \\ Source: https://www.youtube.com/watch?v=K7lWFh3bF0E.}
		\label{fig:productAnalisys}
		\centering
		\includegraphics[width=\textwidth]{figures/productSelection.png}
	\end{center}
\end{figure}

It was clear to me that this graph has not just two, but four dimensions: contribution margin (represented by the y-axis), market share (represented by the x-axis), market size (represented by the size of the circle), and profitability (represented by the color of the circle). \\

In this way, then, the dimensions of our universe can be understood. \\

I believe that we can experience at least eight dimensions of our universe: width, height and depth, time, shades of red, yellow and blue, and light intensity. \\

Maybe it doesn't make much sense, but perhaps it is possible in this way to think in other dimensions.

\end{document}
